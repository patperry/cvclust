\documentclass[11pt]{article}

\usepackage{amsmath}
\usepackage{framed}
\usepackage{fullpage}
\usepackage{hyperref} % must be last usepackage
\usepackage{color}
\usepackage[authoryear]{natbib}

\newcommand{\hilight}[1]{\colorbox{yellow}{#1}}
\newcommand{\todo}{\hilight{TODO}}

\newcommand{\PreserveBackslash}[1]{\let\temp=\\#1\let\\=\temp}
\let\PBS=\PreserveBackslash

\title{``Estimating the number of clusters using cross-validation''
Summary of changes to the revised manuscript}
\author{Wei Fu \and Patrick O. Perry}

\begin{document}

\maketitle
\setlength{\parindent}{0ex}
\setlength{\parskip}{1ex}
\vspace{2ex}

We thank the editors and reviewers for their time reading our submission, and
for providing many helpful suggestions for improvement.  We also thank you for
the opportunity to revise and resubmit our manuscript.  We have revised our
manuscript carefully taking into account your comments.

Detailed responses to reviewer comments follow below.


\section*{Response to Editor}

\begin{itemize}

\item \textit{Fix the figure captions.}

\textbf{Response:} Thank you for the guidance here. Per your instruction, all
figure captions now include (1)~what is the plot about, (2)~specific details
of plot, and (3)~the most important thing that the reader should learn.


\item \textit{Cite R packages.}

\textbf{Response:} We added citations for all packages used in our work: \todo

% You need to cite R and any packages used in your work, like ggplot2.

\item \textit{Reprodicibility.}

\textbf{Response:} \todo

% Your code is quite a mess. It is not clear to me how to reproduce the results
% you have reported. There is code in two different tar.gz files. Which is the
% code to use? In each there are several folders. This all needs to be
% organised so that it is easy for the reader to produce the results that you
% have reported in the paper.


\end{itemize}


\section*{Response to Associate Editor}

\begin{itemize}

\item \textit{Compare to the slope method}

\todo

\end{itemize}


\section*{Response to Reviewer 1}

(No further comments.)


\section*{Response to Reviewer 2}

\begin{itemize}

\item \textit{Bottom of page 4. Clarify meaning of ``fold''.}

\textbf{Response:} Added the following:

``For each value $r$ in the range $1, \dotsc, K$, we define the $r$th `fold'
as the matrix gotten by permuting the rows of the original data matrix to get
$X$, a matrix with the $r$th test subset, the rows $i$ with $r_i = r$ as
its trailing rows.''


\item \textit{Page 6. Clarify that $N \geq K$, $M \geq K$.}

\textbf{Response:} 

Added the sentence: ``For this cross-validation to be possible, we must have that
$K \leq N$ and $L \leq M$.''

%By my understanding, the authors must assume further that
%
%n  >> K
%m >> K	
%
%That is, they need to ensure that they are choosing n and m larger that K.
%This limitation should be clarified to the reader.


\item \textit{Section 2.2. Clarify the context in which the method is applied.}

\textbf{Response:} 

Changed the first sentence as follows: ``Our version of Gabriel
cross-validation for clustering works is designed to find the best number of
clusters to use for $N$~observations of a $P$-dimensional random variable with
real-valued components.''


\item \textit{Section 3. Note that other methods are self consistent.}

\textbf{Response:} Per your suggestion, we added this paragraph:

``Self-consistency by itself does not mean that a procedure performs well in
practice. Self-consistency deals only with the no-noise situation.  Many other
methods for selecting $K$ are likely self-consistent, but not all perform the
same in practice. Self-consistency only suggests (but does not guarantee) that
a procedure might perform reasonably when the noise level is low.''

We did not specifically single out the slope statistic or Silhouette method
specifically here because we have not taken the time to analyze those methods
and we are unaware of proofs of their self-consistency or lack thereof.


%Section 3. I am very confident that the slope statistics (Fujita et al) is
%also self-consistent under the absence of noise. This follows basically from
%the properties of finite differences.  On the other hand, the original
%Silhouette method is not self-consistent. The authors could explore this
%easily in this section in one paragraph.


\item \textit{Proposition 1 improvements.}

\textbf{Response:} We fixed the typo you spotted ($CV(k) > CV(K)$), and we
added a note that the proposition is purely for theoretical purposes:
``The assumptions here are not checkable in practice, but the proposition
suggest that Gabriel cross-validation might give a reasonable answer, at least
when the noise level is low.''

You also suggested that we omit the proposition entirely. We chose to keep it,
but if there are space constraints we will take your suggestion and move the
proof to an appendix.

% 5. Proposition 1. 
% 
% 5.1 In order to check condition (iv), it seems the user must know something about K.
% 
% 5.2 Should it be CV(k)>CV(K)?
% 
% 5.3 My feeling is that once the reader fully understand the ingredients of the
% procedure, Proposition one renders trivial.  The authors could provide direct
% arguments for that. This seems to follow straightforward from the properties
% of estimation in a regression setting.


\item {\textit Proposition 2. Define the indicator function.}

\textbf{Response:} Done.


\item \textit{Page 18, line 14. Generalization conditions.}

\textbf{Response:} 

We clarified what we meant by saying our proof generalized:
\textit{
In principle the proof of Proposition~3 generalizes to
non-normal location families of distributions easily, provided that
both cluster distributions are in the same family and expressions for
$E(Y | Y > 0)$ and the other related quantities are available (and finite).
}

You rightly point out that it will not generalize to all possible situations,
but only to distributions with finite moments and cases where both clusters
have the same distribution.


\item \textit{Figure 2. Please define correctly the x,y axis}

\textbf{Response:} Done.


\item \textit{Page 20, first paragraph.}

\textbf{Response:} Your comment was that ``Proposition 2 only shows that if
$|\rho|<0.5$ the method works correctly.''Did you mean page 14 instead?  We
think you may have meant that our statement in the first paragraph of  page
14, ``Proposition~2 tells us that Gabriel cross-validation fails when there is
strong correlation between the variables'' is too strong.


\item \textit{Page 22, the standardized Euclidean norm $k$-means could produce
good results.}

\textbf{Response:} You may be correct, but we do not want to overwhelm the
reader with two many variations of the same algorithm. The correlation-correction
we propose is a slightly more general version of Euclidean standardization,
and seems to perform adequately.

%10. Page 22, line 3. Maybe the standardized Euclidean norm in the k-means
%algorithm could produce good results. 


\item \textit{Compare to slope}

\textbf{Response:} As you suggested, we added comparisons to the slope method,
using the \texttt{cstab} package. In our simulations, slope performs very well
with noise dimensions and high-dimensional data; its performance degrades with
correlated clusters, variance heterogeneity, and heavy-tailed data. Certainly
there are some situations where slope is better than our method, but no method
uniformly dominates the others. In our simulations, our correlation-corrected
Gabriel CV method seems to be a good default choice.

% 11. Page 23, line 35. Under normal distributions, the slope statistics
% proposed by Fujita et al. also works very good if the clusters are well
% separated. There are two R packages for computing the slope statistics: 
% 
% https://www.rdocumentation.org/packages/anocva/versions/0.1.1/topics/optimalSlope
% 
% and:
% 
% https://cran.r-project.org/web/packages/cstab/cstab.pdf
% 
% 
% 12. Page 25, line 31. The Slope statistic also works well under correlation.
% See the papers simulations (section 5).  Moreover, in Fujita et al., it is
% showed that some of the methods used here have bad performances under some
% scenarios which slope statistic was very good (high dimensional data, t and
% exponential distributions for each cluster, correlated items, Gaussian and
% uniform, heterogeneity). 

\end{itemize}

\end{document}

